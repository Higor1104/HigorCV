\documentclass[10pt, letterpaper]{article}

% Packages:
\usepackage[
    ignoreheadfoot, % set margins without considering header and footer
    top=2 cm, % seperation between body and page edge from the top
    bottom=2 cm, % seperation between body and page edge from the bottom
    left=2 cm, % seperation between body and page edge from the left
    right=2 cm, % seperation between body and page edge from the right
    footskip=1.0 cm, % seperation between body and footer
    % showframe % for debugging 
]{geometry} % for adjusting page geometry
\usepackage{titlesec} % for customizing section titles
\usepackage{tabularx} % for making tables with fixed width columns
\usepackage{array} % tabularx requires this
\usepackage[dvipsnames]{xcolor} % for coloring text
\definecolor{primaryColor}{RGB}{0, 0, 0} % define primary color
\usepackage{enumitem} % for customizing lists
\usepackage{fontawesome5} % for using icons
\usepackage{amsmath} % for math
\usepackage[
    pdftitle={John Doe's CV},
    pdfauthor={Higor Silva},
    pdfcreator={HigorSilva},
    colorlinks=true,
    urlcolor=primaryColor
]{hyperref} % for links, metadata and bookmarks
\usepackage[pscoord]{eso-pic} % for floating text on the page
\usepackage{calc} % for calculating lengths
\usepackage{bookmark} % for bookmarks
\usepackage{lastpage} % for getting the total number of pages
\usepackage{changepage} % for one column entries (adjustwidth environment)
\usepackage{paracol} % for two and three column entries
\usepackage{ifthen} % for conditional statements
\usepackage{needspace} % for avoiding page brake right after the section title
\usepackage{iftex} % check if engine is pdflatex, xetex or luatex

% Ensure that generate pdf is machine readable/ATS parsable:
\ifPDFTeX
    \input{glyphtounicode}
    \pdfgentounicode=1
    \usepackage[T1]{fontenc}
    \usepackage[utf8]{inputenc}
    \usepackage{lmodern}
\fi

\usepackage{charter}

% Some settings:
\raggedright
\AtBeginEnvironment{adjustwidth}{\partopsep0pt} % remove space before adjustwidth environment
\pagestyle{empty} % no header or footer
\setcounter{secnumdepth}{0} % no section numbering
\setlength{\parindent}{0pt} % no indentation
\setlength{\topskip}{0pt} % no top skip
\setlength{\columnsep}{0.15cm} % set column seperation
\pagenumbering{gobble} % no page numbering

\titleformat{\section}{\needspace{4\baselineskip}\bfseries\large}{}{0pt}{}[\vspace{1pt}\titlerule]

\titlespacing{\section}{
    % left space:
    -1pt
}{
    % top space:
    0.3 cm
}{
    % bottom space:
    0.2 cm
} % section title spacing

\renewcommand\labelitemi{$\vcenter{\hbox{\small$\bullet$}}$} % custom bullet points
\newenvironment{highlights}{
    \begin{itemize}[
        topsep=0.10 cm,
        parsep=0.10 cm,
        partopsep=0pt,
        itemsep=0pt,
        leftmargin=0 cm + 10pt
    ]
}{
    \end{itemize}
} % new environment for highlights


\newenvironment{highlightsforbulletentries}{
    \begin{itemize}[
        topsep=0.10 cm,
        parsep=0.10 cm,
        partopsep=0pt,
        itemsep=0pt,
        leftmargin=10pt
    ]
}{
    \end{itemize}
} % new environment for highlights for bullet entries

\newenvironment{onecolentry}{
    \begin{adjustwidth}{
        0 cm + 0.00001 cm
    }{
        0 cm + 0.00001 cm
    }
}{
    \end{adjustwidth}
} % new environment for one column entries

\newenvironment{twocolentry}[2][]{
    \onecolentry
    \def\secondColumn{#2}
    \setcolumnwidth{\fill, 4.5 cm}
    \begin{paracol}{2}
}{
    \switchcolumn \raggedleft \secondColumn
    \end{paracol}
    \endonecolentry
} % new environment for two column entries

\newenvironment{threecolentry}[3][]{
    \onecolentry
    \def\thirdColumn{#3}
    \setcolumnwidth{, \fill, 4.5 cm}
    \begin{paracol}{3}
    {\raggedright #2} \switchcolumn
}{
    \switchcolumn \raggedleft \thirdColumn
    \end{paracol}
    \endonecolentry
} % new environment for three column entries

\newenvironment{header}{
    \setlength{\topsep}{0pt}\par\kern\topsep\centering\linespread{1.5}
}{
    \par\kern\topsep
} % new environment for the header

\newcommand{\placelastupdatedtext}{% \placetextbox{<horizontal pos>}{<vertical pos>}{<stuff>}
  \AddToShipoutPictureFG*{% Add <stuff> to current page foreground
    \put(
        \LenToUnit{\paperwidth-2 cm-0 cm+0.05cm},
        \LenToUnit{\paperheight-1.0 cm}
    ){\vtop{{\null}\makebox[0pt][c]{
        \small\color{gray}\textit{Last updated in September 2024}\hspace{\widthof{Last updated in September 2024}}
    }}}%
  }%
}%

% save the original href command in a new command:
\let\hrefWithoutArrow\href

% new command for external links:


\begin{document}
    \newcommand{\AND}{\unskip
        \cleaders\copy\ANDbox\hskip\wd\ANDbox
        \ignorespaces
    }
    \newsavebox\ANDbox
    \sbox\ANDbox{$|$}

    \begin{header}
        \fontsize{25 pt}{25 pt}\selectfont Higor Araujo Leite e Silva

        \vspace{5 pt}

        \normalsize
        \mbox{Zaventem, Belgium}%
        \kern 5.0 pt%
        \AND%
        \kern 5.0 pt%
        \mbox{\hrefWithoutArrow{mailto:higoraraujosilva@hotmail.com}{higoraraujosilva@hotmail.com}}%
        \kern 5.0 pt%
        \AND%
        \kern 5.0 pt%
        \mbox{\hrefWithoutArrow{tel:+32-047-723-112}{+32-047-723-112}}%
        \kern 5.0 pt%
        \AND%
        \kern 5.0 pt%
        \mbox{\hrefWithoutArrow{https://higor1104.github.io/higorsilva.github.io/}{https://higor1104.github.io/higorsilva.github.io/}}%
        \kern 5.0 pt%
        \AND%
        \kern 5.0 pt%
        \mbox{\hrefWithoutArrow{https://linkedin.com/in/higoraraujosilva}{linkedin.com/in/higoraraujosilva}}%
        \kern 5.0 pt%
        \AND%
        \kern 5.0 pt%
        \mbox{\hrefWithoutArrow{https://github.com/higor1104}{github.com/higor1104}}%
    \end{header}

    \vspace{5 pt - 0.3 cm}



    \section{Introdução de Carreira}

        \begin{onecolentry}
            Desenvolvedor Java Sênior com ampla experiência em projetar e modernizar aplicações escaláveis e seguras. Forte especialização em Java, no ecossistema Spring e microserviços, com uma sólida formação em DevOps, plataformas de nuvem e gestão de banco de dados. Habilidade em migrar sistemas legados para soluções conteinerizadas e prontas para a nuvem. Atualmente expandindo meus conhecimentos em Quarkus e explorando casos práticos de uso. Certificado em Java e práticas ágeis, fluente em inglês e português com francês e holandês básicos, com experiência profissional no Brasil, Portugal e Bélgica.
        \end{onecolentry}

        \vspace{0.2 cm}


    \section{Educação}

        \begin{twocolentry}{
            Fev 2011 – Mai 2016
        }
            \textbf{Universidade do Cotemig}, Bacharel em Sistemas de Informação\end{twocolentry}

        \vspace{0.10 cm}

        \begin{onecolentry}
            \begin{highlights}
                \item \textbf{Disciplinas:} Desenvolvimento de websites, apresentando todas as fases de entrega desde o conceito ideal de um cliente até a documentação e apresentação para o "cliente"
            \end{highlights}
        \end{onecolentry}

        \vspace{0.15 cm}
        \begin{twocolentry}{
            Mai 2015
        }
            \textbf{ORACLE}, Oracle Certified Associate, Programador Java SE 7
        \end{twocolentry}
            
        \vspace{0.15 cm}

        \begin{twocolentry}{
            Jul 2009 – Dez 2010
        }
            \textbf{Formação Profissional Senai}, Curso Técnico
        \end{twocolentry}

        \vspace{0.10 cm}

        \begin{onecolentry}
            \begin{highlights}
                \item \textbf{Disciplinas:} Conceitos de software e hardware, manutenção de computadores, servidores Windows e Linux, conceitos de eletricidade
            \end{highlights}
        \end{onecolentry}

    \section{Experiência}

        \begin{twocolentry}{
            Fev 2024 – Atual
        }
            \textbf{Engenheiro de Software}, EEAS -- Bruxelas, BE
        \end{twocolentry}

        \vspace{0.10 cm}
        
        \begin{onecolentry}
            \begin{highlights}
                \item Migrei várias aplicações para uma arquitetura DevOps utilizando Docker, JDK 21, VueJS e Spring Boot 3, GitHub Actions e Tenzur
                \item Refatorei lógica complexa, eliminei cheiros de código e implementei testes unitários com Mockito, garantindo a qualidade e manutenção do código
                \item Projetei e implementei uma nova camada de segurança usando Spring Security e OIDC, melhorando os padrões de autenticação e autorização
                \item Desenvolvi novas funcionalidades e correções de bugs, seguindo as melhores práticas da indústria para softwares escaláveis e fáceis de manter
                \item Monitorei e resolvi problemas de aplicações utilizando Dynatrace, garantindo alta confiabilidade e desempenho
            \end{highlights}
        \end{onecolentry}
        
        \vspace{0.2 cm}

        \begin{twocolentry}{
            Jul 2022 – Nov 2023
        }
            \textbf{Desenvolvedor Back-end}, ING Bank -- Bruxelas, BE
        \end{twocolentry}

        \vspace{0.10 cm}
        
        \begin{onecolentry}
            \begin{highlights}
                \item Apoiei e migrei uma aplicação legada Java EE para uma arquitetura moderna containerizada DevOps com Spring Boot, OpenShift, melhorando escalabilidade, manutenção e desempenho
                \item Analisei requisitos e implementei novas funcionalidades e melhorias para sistemas legados e modernizados
                \item Mantive aplicações legadas em servidores WebSphere enquanto fazia a transição para um ambiente de microsserviços com containerização utilizando Red Hat OpenShift
                \item Colaborei em uma equipe ágil, garantindo a entrega pontual e de alta qualidade das soluções
                \item Monitorei aplicações e desempenho utilizando Kibana, Grafana e Prometheus, possibilitando a resolução proativa de problemas e confiabilidade do sistema
            \end{highlights}
        \end{onecolentry}
        
        \vspace{0.2 cm}

        \begin{twocolentry}{
            Abr 2021 – Jun 2022
        }
            \textbf{Desenvolvedor Back-end}, Waitrose -- Lisboa, PT
        \end{twocolentry}

        \vspace{0.10 cm}
        
        \begin{onecolentry}
            \begin{highlights}
                \item Contribuí para a migração de aplicações legadas de varejo para uma plataforma moderna e escalável utilizando tecnologias de ponta, suportando serviços online, sistemas em loja, autoatendimento e integrações de terceiros.
                \item Projetei e implementei APIs REST seguras e de alto desempenho com robusta autenticação, autorização e monitoramento de desempenho, seguindo os padrões da empresa.
                \item Apliquei Programação Reativa com o Spring Project Reactor para melhorar a resposta do sistema e resiliência sob alto tráfego.
                \item Desenvolvi microsserviços utilizando Spring Boot e Spring Cloud, implantados em infraestrutura AWS (\textbf{Lambda, SNS, SQS}, RDS, KMS, CloudFormation) para garantir escalabilidade e confiabilidade.
                \item Observei sistemas de produção e resolvi incidentes, utilizando Grafana e a stack ELK para observabilidade e gerenciamento proativo de problemas.
            \end{highlights}
        \end{onecolentry}
        
        \vspace{0.2 cm}

        \begin{twocolentry}{
            Set 2018 – Abr 2021
        }
            \textbf{Engenheiro de Software}, BMG Bank -- Minas Gerais, BR
        \end{twocolentry}

        \vspace{0.10 cm}
        
        \begin{onecolentry}
            \begin{highlights}
                \item Apoiei uma das maiores plataformas de crédito do país, acessada por mais de 40 mil usuários diários e processando mais de 100 mil propostas de crédito, seguros, empréstimos e criação de contas
                \item Participei de discussões de negócios para traduzir requisitos em novas funcionalidades e soluções técnicas
                \item Projetei e desenvolvi microsserviços e aplicações Java EE/Spring Boot, melhorando escalabilidade e entrega de funcionalidades
                \item Implementei a stack ELK para gerenciamento centralizado de logs, facilitando a solução de problemas e o monitoramento
                \item Colaborei com desenvolvedores para garantir a qualidade do código e a entrega confiável em um ambiente ágil
            \end{highlights}
        \end{onecolentry}

        \vspace{0.2 cm}

        \begin{twocolentry}{
            Jun 2018 – Ago 2018
        }
            \textbf{Engenheiro de Software}, CI\&T (Projeto iHeartMedia) -- Minas Gerais, BR
        \end{twocolentry}

        \vspace{0.10 cm}
        
        \begin{onecolentry}
            \begin{highlights}
                \item Contribuí para a migração de múltiplos sistemas legados para uma arquitetura de microsserviços resiliente, melhorando escalabilidade, modularidade e manutenção.
                \item Desenvolvi APIs RESTful com Java 8, Spring Framework e Hibernate, que alimentam os serviços back-end de vários sites de alto tráfego.
                \item Projetei e implementei esquemas NoSQL no Cassandra, permitindo consultas rápidas em grandes volumes de dados e garantindo estabilidade sob alta carga.
                \item Colaborei com equipes distribuídas (Scrum com equipe dos EUA, OnePieceFlow localmente), garantindo entregas ágeis e confiáveis.
                \item Utilize tecnologias modernas, como a stack Spring Netflix (Ribbon, Eureka, Feign, Hystrix), Kubernetes, Gradle e ELK Stack para monitoramento e resiliência.
            \end{highlights}
        \end{onecolentry}

        \vspace{0.2 cm}

        \begin{twocolentry}{
            Abr 2017 – Mai 2018
        }
            \textbf{Engenheiro de Software}, BMG Bank -- Minas Gerais, BR
        \end{twocolentry}

        \vspace{0.10 cm}
        
        \begin{onecolentry}
            \begin{highlights}
                \item Entreguei correções de código cruciais em todas as camadas de uma aplicação web financeira em larga escala
                \item Melhorei sistemas legados para suportar novos requisitos de negócios e usuários
                \item Resolvi problemas de arquitetura, reduzindo o tempo de desenvolvimento, melhorando a qualidade do código e aumentando a manutenção
                \item Colaborei em um ambiente ágil (Scrum), apoiando CI/CD com Jenkins e monitoramento de desempenho com CA Introscope
            \end{highlights}
        \end{onecolentry}
        
        \vspace{0.2 cm}
        

        \begin{twocolentry}{
            Fev 2016 – Mar 2017
        }
            \textbf{Desenvolvedor Full Stack}, IMedicina -- Minas Gerais, BR
        \end{twocolentry}

        \vspace{0.10 cm}
        
        \begin{onecolentry}
            \begin{highlights}
                \item Contribuí para uma plataforma nacional de gestão médica, apoiando o atendimento aos pacientes, alocação de recursos e operações financeiras
                \item Gerenciei instâncias de aplicativos Docker e apoiei CI/CD com Jenkins
                \item Implementei segurança baseada em OAuth2 para comunicação entre microsserviços
                \item Resolvi bugs e problemas técnicos, garantindo a confiabilidade do sistema e entregas pontuais sob pressão
                \item Trabalhei com uma stack de tecnologias moderna, incluindo Spring, Hibernate, QueryDSL, APIs REST, AngularJS, MySQL, Neo4J, Apache Solr e Linux
            \end{highlights}
        \end{onecolentry}
        
        \vspace{0.2 cm}
        
        \begin{twocolentry}{
            Jun 2013 – Jan 2016
        }
            \textbf{Desenvolvedor Java}, Cast IT Group -- Minas Gerais, BR
        \end{twocolentry}

        \vspace{0.10 cm}
        \begin{onecolentry}
            \begin{highlights}
                \item Desenvolvi e mantive aplicativos web (AngularJS, JSF, Apache Wicket) e desktop (Java Swing)
                \item Implementei soluções de segurança, persistência e build utilizando Spring Security, Hibernate, Maven e JBoss
                \item Gerenciei bancos de dados (Oracle, MySQL) e apliquei controle de versão com SVN/TortoiseSVN
                \item Participei de programas de treinamento e certificação (Java, C-Sharp, Scrum)
            \end{highlights}
        \end{onecolentry}
        
    \section{Estudos Pessoais}

        \vspace{0.10 cm}
        \begin{onecolentry}
            \begin{highlights}
                \item Atualmente expandindo minha expertise em Quarkus, explorando casos práticos de uso e experimentando com aplicações reais. Buscando oportunidades para aplicar e compartilhar esse conhecimento em projetos profissionais, contribuindo para soluções inovadoras e de alto desempenho.
            \end{highlights}
        \end{onecolentry}
        
    \section{Habilidades}

        \begin{onecolentry} \textbf{Habilidades interpessoais}\end{onecolentry}

        \vspace{0.10 cm}

        \begin{onecolentry}
            \begin{highlights}
                \item Colaboração com equipes distribuídas e multiculturais (Brasil, Portugal, Bélgica, EUA)
                \item Fortes habilidades de resolução de problemas e análise sob pressão
                \item Comunicação eficaz com stakeholders de negócios e técnicos
                \item Adaptabilidade e aprendizado rápido de novas tecnologias
                \item Mentoria e orientação de equipes sobre melhores práticas e qualidade de código
            \end{highlights}
        \end{onecolentry}

        \vspace{0.2 cm}

        \begin{onecolentry} \textbf{Habilidades técnicas}\end{onecolentry}

        \vspace{0.10 cm}

        \begin{onecolentry}
            \begin{highlights}
                \item \textbf{Programação \& Frameworks} - Java 11–21, Spring Boot 3, Spring Security, Microsserviços, Programação Reativa (Spring Project Reactor), APIs RESTful, Angular
                \item \textbf{Cloud \& Containerização} - AWS (Lambda, SNS, SQS, RDS, KMS, CloudFormation), Microsoft Azure, Docker, Kubernetes, OpenShift
                \item \textbf{Bancos de Dados \& Gerenciamento de Dados} - MySQL, Oracle, Redis, Cassandra (NoSQL), Neo4J
                \item \textbf{DevOps \& Observabilidade} - CI/CD (Jenkins, GitLab, GitHub Actions, Azure Devops), Gradle, Maven, ELK Stack, Grafana, Dynatrace, SonarQube
                \item \textbf{Práticas Modernas} - Ágil (Scrum/Kanban), Microsserviços Contenerizados, Melhores Práticas de Segurança, Arquitetura Reativa \& Orientada a Mensagens
            \end{highlights}
        \end{onecolentry}

\end{document}
