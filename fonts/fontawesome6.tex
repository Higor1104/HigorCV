\documentclass{scrartcl}
\usepackage{hyperref}
\usepackage{shortvrb}
\usepackage{metalogo}
\usepackage{longtable}
\usepackage{fontawesome6}
% \usepackage{xcolor}
% \usepackage[pro]{fontawesome6}
% \faStyle{duotone-solid}
\usepackage[utf8]{inputenc}
\usepackage{geometry}
\MakeShortVerb{\|}
\usepackage[english]{babel}
\begin{document}
\title{The \texttt{fontawesome6} Package\thanks{This document corresponds to fontawesome6 version 6.7.2-3, dated 2025/09/16.}}
\author{Font Awesome\thanks{More information at \url{https://fontawesome.com}} (The font)\and Daniel Nagel\thanks{GitHub: \href{https://github.com/braniii}{github.com/braniii}} (The \LaTeX{} package)}
\maketitle

This package provides \LaTeX{} support for Font Awesome 6 icons.

Special thanks to Marcel Krüger for the original \texttt{fontawesome5} package\\
(\url{https://ctan.org/pkg/fontawesome5}), upon which this package is based.

\subsection*{Usage}

To use Font Awesome 6 icons in your document, load the package with:
\begin{verbatim}
  \usepackage{fontawesome6}
\end{verbatim}
Optionally, you can add the |fixed| option to enable fixed-width icons:
\begin{verbatim}
  \usepackage[fixed]{fontawesome6}
\end{verbatim}

Each icon is available as a macro, using the official icon name in CamelCase with the prefix |\fa|.\\
For example, to use the |hand-point-up| icon, write |\faHandPointUp|.\\
An optional argument allows you to select the style (|solid| or |regular|). The
default style is |solid|, but you can change it globally with |\faStyle{...}| or
by setting the option |style=...|.

Alternatively, you can access any icon by its official name using |\faIcon{icon-name}| or |\faIcon[style]{icon-name}|.

A comprehensive list of all included icons and their corresponding commands is provided at the end of this document.

\subsection*{Example}
\begin{verbatim}
...
\usepackage{fontawesome6}
...
\begin{document}
...
A simple icon: \faHandPointUp\\
Multiple versions of the file icon:
  \faFile~
  \faFile[solid]~
  \faFile[regular]~.\\
Alternative syntax:
  \faIcon{file}~
  \faIcon[solid]{file}~
  \faIcon[regular]{file}~.
...
\end{document}
\end{verbatim}

A simple icon: \faHandPointUp\\
Multiple versions of the file icon: \faFile~\faFile[solid]~\faFile[regular].\\
Alternative syntax: \faIcon{file}~\faIcon[solid]{file}~\faIcon[regular]{file}.

\subsection*{Font Awesome Pro}
Font Awesome 6 is available in both Free and Pro versions. By default, this package uses the Free version. If you have a Pro license and have installed the Font Awesome 6 Pro desktop fonts in your system font path, you can enable Pro support by loading the package with the |[pro]| option:
\begin{verbatim}
  \usepackage[pro]{fontawesome6}
\end{verbatim}

With Pro enabled, the following additional styles are available: |solid|, |regular|, |light|, |thin|, |duotone-solid|, |duotone-regular|, |duotone-light|, |duotone-thin|, |sharp-solid|, |sharp-regular|, |sharp-light|, |sharp-thin|, |sharp-duotone-solid|, |sharp-duotone-regular|, |sharp-duotone-light|, and |sharp-duotone-thin|.

For duotone icons, you can set the secondary color using |\faDuotoneSetSecondary|:
\begin{verbatim}
  % Remember to load xcolor
  % Set secondary color to green
  \faDuotoneSetSecondary{\color{green}}
\end{verbatim}
Pro features are supported only with \XeLaTeX{} and \LuaLaTeX.

\subsection*{Updates}
This package corresponds to Font Awesome 6.7.2.\\
If a newer version is available on the Font Awesome website, check for updates at \url{https://ctan.org/pkg/fontawesome6}. If the latest version is not yet on CTAN, you may contact \href{mailto:tex@2krueger.de}{\nolinkurl{tex@2krueger.de}}.

If you use \XeLaTeX{} or \LuaLaTeX{}, you can also manually download the new Desktop Fonts from \url{https://fontawesome.com} and place them in your \TeX{} tree. Save them with the following filenames:
{\ttfamily
\begin{tabular}{l}
  FontAwesome6Brands-Regular-400.otf\\
  FontAwesome6Free-Regular-400.otf\\
  FontAwesome6Free-Solid-900.otf
\end{tabular}
}\\
The package will then automatically use the new version.

\subsection*{Bugs and Feedback}
For bug reports or feature requests, please open an issue at \href{https://github.com/braniii/fontawesome}{github.com/braniii/fontawesome}.

\input{fa6_fulllist}
\end{document}
